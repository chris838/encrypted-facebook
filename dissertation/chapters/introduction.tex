%************************************************
\chapter{Introduction}\label{ch:introduction}
%************************************************

Facebook, at the time of writing, is the world's most popular social network service with over 500 million users active in the last 30 days \cite{fb-factsheet}. This project describes Encrypted Facebook, a Firefox extension which aims to protect privacy by applying broadcast encryption to content shared through the Facebook platform. As well as textual content, JPEG compression immune coding is used to support encrypted images.

\section{Background}
\label{sec:background}

Since its inception, Facebook's weak and often seemingly indifferent approach to on-line privacy has come under criticism \cite{fb-cipc}. As social networks mimic real life interactions, members are inclined to reveal more private details than they otherwise would, resulting in the accumulation of a large repository of sensitive information \cite{gross}. There are a number of ways this information can then be exposed: users misconfiguring privacy settings,\footnote{Facebook was recently forced to update its privacy controls due to growing pressure from the public and media \cite{fb-priv}.} malice, error or neglect on the part of the social network\footnote{As was the case in 2010 when personally identifiable Facebook user details were sold to data brokers by third party developers \cite{fb-ids}.} and its employees \cite{snoop}, acquiescent disclosure to government authorities \cite{fb-gov} and unlawful access through phishing scams \cite{fb-phish} and security exploits \cite{rockyou}.

Encrypting content ensures its secrecy in any eventuality, provided the encryption key itself is kept safe. Tools do exist for manually encrypting online exchanges \cite{firegpg}. However, studies have suggested that software which requires manual management of cryptographic keys is not usable enough to provide effective security for most users \cite{johnny}. In addition, interaction over social networks is typically one-to-many, unlike applications like e-mail which these tools are designed for.

Solutions targetting social networks specifically have been proposed. Some are still overly reliant on manual key management; others fail to fully protect the user's privacy or support the most popular forms of content (see sections \ref{sec:existing}, \ref{sec:principles} and \ref{sec:facebook}). Thus far all proposals have required the use of a third party in ways which are unlikely to be scaleable.
  
Attempts have been made at creating new social networks which use encryption and decentralised hosting to provide better privacy protection \cite{pidder}\cite{diaspora}. Unfortunately, network externalities make it difficult for newcomers to compete with Facebook since the utility of a social network is intrinsically tied to the size of its userbase. \footnote{Some suggest the value of a social network grows linearithmically, quadratically or even exponentially with the number of users \cite{fb-network} \cite{metcalf}.}


\section{Objectives}
\label{sec:objectives}

The aim of this project is to provide an enhanced privacy solution for existing Facebook users which is:

\begin{itemize}

    \item \textbf{Privacy preserving} given the scenarios presented in Section \ref{sec:background}.

    \item \textbf{Useable} and therefore accessible to the typical Facebook user.

    \item \textbf{Scaleable} given the vast size of Facebook's userbase. 
    
    \item \textbf{Incrementally deployable} in order to avoid some of the problems of network effects.

\end{itemize}


\section{Limitations}
\label{sec:limit}

Some objectives we consider non-goals, or outside the scope of the project:

\begin{itemize}

    \item Securing the structure of the social graph itself.
    
    \item Ensuring the integrity, authenticity or non-repudiation of communications. In theory the public key scheme could be extended to do so but this would go beyond mere privacy control.

    \item Complete protection against middleperson attacks (see Section \ref{ssec:threat}).
    
    \item Guaranteeing availability of content. Completely severing reliance on Facebook would be tantamount to building a new network.
    
    \item Concealing the existence of content itself or concealing the fact that it is encrypted. Steganography is employed but not for this reason (see Section \ref{ssec:submitnote}).
    
    \item Full cross-platform and/or cross-browser support (see Section \ref{sec:future}).
    
\end{itemize}


\section{Existing work}
\label{sec:existing}

    Many applications for generally encrypting online data exist, most in the form of browser extensions \cite{firegpg}. There are also several applications targeting Facebook specifically which we introduce below. Section \ref{sec:principles} describes how they relate to our own approach.
    
    \begin{desc}
    
        \item[uProtect.it] Client side JavaScript application which inserts UI controls and intercepts content. Content is encrypted, stored and decrypted all on a third party server. \hfill \\
        \url{http://uprotect.it}
        
        \item[FaceCloak] Firefox extension which posts fake content to Facebook, using it as an index to the encrypted content on a third party server. Running your own server is possible, though only users on the same server can communicate. \hfill \\
        \url{http://crysp.uwaterloo.ca/software/facecloak/}
        
        \item[flyByNight] Content is submitted through a Facebook application, encrypted using client side JavaScript and passed via Facebook to a third party application server. Proxy re-encryption is used for sending to multiple recipients. \hfill \\
        \url{http://hatswitch.org/~nikita/}
        
        \item[NOYB] Content is stored in plaintext but profiles are anonymised. The mapping from real-to-fake profiles is known only to a user's friends. \hfill \\
        \url{http://adresearch.mpi-sws.org/noyb.html}
        
    \end{desc}






