%************************************************
\chapter{Introduction}\label{ch:introduction}
%************************************************

Facebook is, at the time of writing, the world's most popular social network service, with over 500 million users active in the last 30 days \cite{fb-factsheet}. Currently all communication on Facebook is in plaintext. This project implements a Firefox extension which adds a broadcast encryption layer over the Facebook interface without significantly impacting the typical Facebook user experience.

\section{Background}

Since its inception, Facebook has come under criticism in relation to online privacy \cite{fb-cipc}.

General context - recent media coverage of Facebook, privacy and security concerns. Data (user passwords) stolen from web sites by hackers. Privacy a big issue for the 21st century?
  
  
  
Problems with confusing users and complicated interfaces (e.g. Facebook defaults). Problems with intrusiveness (e.g. SSL and Google, Facebook).

 Clearly any privacy/security solution must be unobtrusive and transparent. In 2010, pressure from users and the media eventually lead to Facebook themselves streamlining their privacy controls \cite{fb-priv} to make them simpler and more intuitive.
  
  
Recently there have been attempts to create alternatives to Facebook \cite{pidder} which provide better protection for user privacy by encrypting shared content. Some propose to go further \cite{diaspora} and decentralize hosting of the social network platform. Unfortunately, network externalities make it very difficult to compete with an established social network service \cite{fb-network}. 

Here we avoid this problem by providing enhanced privacy for existing Facebook users. By adding a broadcast encryption layer on top of the Facebook social networking service we can ensure confidentiality of shared content to a select set of recipients.

\section{Limitations}

What the application does not do:
\begin{itemize}
    \item Does not hide the social graph. Arguably this is Facebook's real asset. But that's another story...
    \item Does not ensure integrity of data. Facebook employees could swap your messages.
    \item Does not ensure availability. Facebook could easily wipe your notes or images.
    \item Does not ensure authenticity or non-repudiation. Not from the employees of Facebook anyway.
    \item Threat model not comprehensive due to project limitations. You can turn off images, which haven't been audited.
\end{itemize}


\section{Existing work}
\begin{itemize}
    \item Symmetric only schemes:
    
    \begin{itemize}
        \item FireGPG - \url{http://blog.fortinet.com/encrypting-facebook/}
        \item CryptFire
        \item TextCrypt - \url{http://subrosasoft.com/OSXSoftware/index.php?main_page=product_info&cPath=210&products_id=207}
    \end{itemize}
    
    Lots of these around.
    
    \item Complete schemes:
    
    \begin{itemize}
        \item uProtect.it
        \item flyByNight - \url{http://hatswitch.org/~nikita/papers/flybynight.pdf}
    \end{itemize}
\end{itemize}




