%*****************************************
\chapter{Conclusion}\label{ch:conclusion}
%*****************************************
\label{con}

To conclude, we evaluate the project against the requirements in Section \ref{sec:req} and take a retrospective look at what would have been done differently given the benefit of hindsight. We finish with a discussion of the possible barriers to potential deployment and future work that might be relevant.

\section{Evaluation of Requirements}
\label{sec:eval-require}

We believe that all the requirements described in Section \ref{sec:req} were met in full:

\begin{desc}

    \item[Requirement 1] The extensions will be able to broadcast-encrypt, submit, retrieve and decipher the following objects:
    
    \begin{itemize}
        \item Status updates
        \item Wall posts
        \item Comments
        \item Messages
        \item Images
    \end{itemize}
    
    \item[Defence] Submission and retrieval of comments and images is demonstrated in Section \ref{sec:cw}, and of status updates in Section \ref{sec:ptextc}. Section \ref{sec:parsing} describes how other encrypted items and submission controls are supported.
    
    \item[Requirement 2] The size limits for encrypted text objects will be no smaller than the current limits Facebook imposes, given in Section \ref{sec:content}).
    
    \item[Defence] Section \ref{sec:recsize} demonstrates submitting and decoding textual content 10,000 characters (the largest limit) in length. Since all text content is stored in notes, the other limits in Section \ref{sec:content} are artificially imposed during sanitisation.

    \item[Requirement 3] The size limit for encrypted images will be no less than 50 KiB.
    
    \item[Defence] Section \ref{sec:recsize} and Section \ref{sec:ptextc} both demonstrate submitting and retrieving images of 50 KiB in length.

    \item[Requirement 4] All requirements will be met given recipient groups sizes up to 400, based on the discussion in Section \ref{sec:cness}.
    
    \item[Defence] All the examples given in this section use recipient group sizes of 400 where applicable.
    
    \item[Requirement 5] All response times will lie within acceptable limits, in accordance with \cite{response}.
    
    \item[Defence] All response times in Section \ref{sec:ptextc} are under 10 seconds --- the recommended limit for response times with no progress indicator. The one exception is the page load time for an entire news feed of non-cached encrypted images. Section \ref{sec:ptextc} shows in this case the time between successive loads is less than one second on average. Intermittent loading of images may be considered an acceptable progress indicator \cite{response}.

    \item[Requirement 6] The project will adhere to the security policy described in \ref{ssec:threat}.
    
    \item[Defence] All results presented in Chapter \ref{eval} demonstrate the use of 256-bit and 2048 bit key lengths for AES and RSA respectively, which exceeds the required minimum (128-bit AES and 2048-bit RSA). Sections \ref{ssec:over}, \ref{ssec:ident-targets}, \ref{ssec:proc-targets} and \ref{ssec:mankeys} describe the implementation of remaining security measures.

    \item[Requirement 7] The number of news feed entries generated by encrypted submission will be no more than the number generated by normal content submission.
    
    \item[Defence] Section \ref{sec:ptextc} relies on the fact that only encrypted items themselves are posted to the news feed. Section \ref{ssec:submitnote} details the delete mechanism used to ensure this is the case.
    

    \item[Requirement 8] The application will adopt a modular structure that facilitates switching between different schemes and permits future extension.
    
    \item[Defence] Section \ref{sec:modstruct} describes the abstract factory pattern used for library sub-components. Sections \ref{sec:imgcod} demonstrates an example of polymorphic use of the IConduitImage sub-component.
    
\end{desc}



\section{Retrospective}
\label{sec:retro}

In retrospect is would have been better not to implement the HWT coding method due to its poor capacity. One alternative might have been to explore encoding data in the low frequency coefficients of a lossless DCT-like transform such as that used in the upcoming JPEG XR format \cite{jpegxr}.

Whilst evaluating capacity (Section \ref{sec:imgcod}) provisional tests using multiple library instances were performed. Inspection of CPU utilisation and timing results suggests that parallelisation of the encoding/decoding process would offer considerable speed improvements. Ideally multi-threading capability should have been included as a initial requirement since this might be difficult to retrofit.

Facebook also recently increased their image size limits. If this were known in advance it would have had a considerable effect on image coding development and evaluation.\footnote{Discussed in detail in Appendix \ref{app:imginc}.}

\section{Barriers to deployment}
\label{sec:deploy}

Several issues need to be overcome before the extension is deployment-ready.

Given the network effects involved, any system needs to be available for as many platforms as possible. Currently the extension is only tested on Linux, though Firefox extensions are generally well suited to cross platform deployment. With regard to deployment on multiple browsers, the structure presented in Section \ref{ssec:over} means that while both the Toolbar XUL and large parts of the {\tt efb} module would have to be re-developed, the remainder of the project could be used largely unmodified.

Network effects also mean that splitting the user base would be highly undesirable. Ideally the extension should not only support a modular, extensible structure, but also backwards compatibility between newer and older versions. In particular this would require a canonical JPEG-immune signature that describes what modules were used to encode an image. One possibility would be to use an n-bit scaling method with $n=1$.

The Botan, CImg and Schifra library components are distributed under the FreeBSD, CeCILL and GPLv2\footnote{Provided the use is non-profit.} licenses respectively, which are all GPL compatible \cite{gpl}. We would have to make sure the terms of the GPL are adhered to.


\section{Future work}
\label{sec:future}

A more advanced broadcast encryption scheme is presented in Appendix \ref{app:bcast}, partly based on Boneh et al \cite{boneh}. In brief, the scheme involves appending transmission overheads to the user's public key.\footnote{This would involve building a data structure to store the public key, most likely a linked list of Facebook notes.} Session keys can then be re-used when transmitting to the same recipient group. The key expiration period\footnote{NIST recommendations for key re-use periods range from a few months up to two years, depending on the nature of the message. \cite{nist-key}} could provide a sliding parameter that trades lower amortised storage overheads for better security. In addition, by running clean-up routines periodically on the public key, old keys can be deleted both to save space and to give content a finite lifetime.

Finding an optimal approach to high capacity JPEG-immune image coding would be an interesting research-level problem. One promising avenue might be the use of a dirty paper coding scheme \cite{dpaper}.







