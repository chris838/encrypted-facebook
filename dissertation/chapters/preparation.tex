%*****************************************
\chapter{Preparation}\label{ch:preparation}
%*****************************************

\section{Approaches}
    \begin{itemize}
        
        \item First, assume all communication with Facebook and any third party server is done securely. Can't we just store message on a server, authenticate and server will let you download? No. This just defers the problem (uProtect.it does this). We actually want cryptographicly secure communication - if the server contents were made public we would still be OK.
        
        \item Choices we have, based mainly on existing work:
        \begin{itemize}
        
        \item Where are the cryptographic functions executed?
            \begin{itemize}
                \item On the client computer
                \item On a remote server (uProtect.it do this. If we perform encryption/decryption remotely we place trust in the remote party. Bad bad bad).
            \end{itemize}
                
        \item If any code is run locally, by what mechanism do we obtain it?
            \begin{itemize}
                \item JavaScript downloaded from a site (uProtect, FB apps) every time we run.
                \item Downloaded and installed once, maybe from trusted source (package manager, Firefox extension library, App store etc.)
            \end{itemize}
                
        \item Where are the private keys stored?
            \begin{itemize}
                \item Not at all (relies on memorised passwords). Can't have big enough keyspace.
                \item Locally (requires performing decryption locally or sending keys over a secure connection). 
                \item Remotely (requires performing decryption remotely or recieving keys over a secure connection). More portable but less secure.
            \end{itemize}
                
        \item Where are public keys stored and distrubuted?
            \begin{itemize}
                \item Locally, and distributed manually. Not really useable.
                \item Through/from a trusted keychain.
                \item With a Facebook or other third party.
            \end{itemize}
                
        \item Where do we store messages?
            \begin{itemize}
                \item Locally isn't feasible
                \item On Facebook
                \item On a third party server.
            \end{itemize}
                
        \item Where do we intercept Facebook interaction? Show a nice diagram.
            \begin{itemize}
                \item Behind the browser (remotely) (e.g. a remotely hosted Facebook client running server side code)
                \item Behind the browser (locally) (e.g. proxy server on localhost)
                \item In the browser (inside the sandbox) (e.g. JavaScript, Java Applet (ughh))
                \item In the browser (outside the sandbox) (e.g. Greasemonkey or extension. Maybe Flash aswell since can access local filesystem but Chrome sandboxes \url{http://blog.chromium.org/2010/12/rolling-out-sandbox-for-adobe-flash.html})
                \item No browser - custom built client
            \end{itemize}
        \end{itemize}
        
	\item The conclusion:
            \begin{itemize}
                \item Minimal third party reliance - not much benefit since trust is just defered. Plus downsides - can't scale, reliability, complexity. Everything stored/performed on Facebook or locally. One exception is app itself.
                \item If we want to store local state i.e. private keys, we can't use remote client, Greasemonkey or sandbox browser stuff. Custom client and proxy server just too complex. So extension all thats left.
                \item Due to JavaScript being shit we need some native code. Only choices are Python or C/C++. Speed is in our criteria. Do a side by side speed comparison (and of JavaScript as well). C is faster => we use C.
            \end{itemize}
        
	\item How to call C++ from JavaScript (binding, linking, compilation, marshalling). You should probably find out how this works. Note that we need Gecko 2.0 so we can avoid all that XPCOM nonsense.
                
    \end{itemize}
    
\section{Facebook}

\begin{itemize}
   
    \item What the average Facebook user does \url{http://www.onlineschools.org/blog/facebook-obsession/}. Comments by far the most common activity. Messages and photos next most common. Then friends  but this is social graph. Then status updates, wall posts. Invites and tags - again social graph. Finally links, not social graph but hard to integrate - doesn't matter too much since less popular than other usage forms. Can still share links manually anyway. Likes are propably on there somewhere - again we don't care.
    
    \item Some limitations on objects: 
    \begin{itemize}
        \item Status update, 144 chars
        \item Post
        \item Comment
        \item Note, 66,000 chars or so
        \item Image, 720x720 3 channel 8 bit colour. Not really because of chrominance subsampling.
    \end{itemize}
    Clearly images and notes are the most information dense so we use these.
    
    \item Connectedness.
    \begin{itemize}
        \item Facebook says average is 130. This paper \url{http://arxiv.org/PS_cache/cs/pdf/0611/0611137v1.pdf} says average is 170 and distrubtion drops sharply at 250. Cameron Marlow says we only speak to a core group of friends anyway.
        \item Theoretical limits - Dunbar number of 150, others suggest higher at ~300. So it perhaps won't increase with Facebook's expansion.
        \item Conclusion - 400 if we can make it work, but anything as low as 100 we could probably get away with.
    \end{itemize}
    
    \item Interacting with Facebook. Facebook Graph API (JavaScript API etc.). What we can do. What we can't that we need to (writing to profiles, problems with album ids etc.) Why I didn't use the JavaScript SDK - poorly documented, working with images is a pain. For the workarounds - iframes and forms - see the implementation.

    \item Notes on application deployment. Need to allow for INCREMENTAL DEPLOYMENT. Needs to be an opt in scheme, don't want to divide SNS. Non-users must be able to co-exist with users.


\end{itemize}
    
\section{Storing data in images}
    Images are the big thing. uProtect.it don't do images, FlyByNight talk about it as an extension.
    \begin{itemize}
	\item Description of Facebook/JPEG compression process (problem statement)
	\item Analysis of theoretical capacity
	\item Evaluation of naive implementations (MATLAB demonstrations)
            \begin{itemize}
		\item Completely naive (encode in RGB values)
		\item Slightly less so (encode in DST with bit masks)
                \item Preliminary testing with Haar wavelets (but no lifting scheme)
                \item Optimal scheme? Dirty paper coding or similair? Its just too hard a problem but future work could be interesting.
             \end{itemize}
	\item Conclusion: possible approaches : Haar and Scale. Modularity and extensibility useful though since we don't know which is best and both are sub-optimal.
    \end{itemize}
    
\section{Encryption schemes}
    \begin{itemize}
        \item Threat model/analysis.
        \item Proxy re-encryption, like FlyByNite. Requires server side encryption so we leave it.
        \item Broadcast encryption.
        \begin{itemize}
            \item Naive implementation.
            \item More advanced schemes (and why I don't use them). Mainly due to no server side code, can't ask users to perform operations and expect a reply any time soon (or at all). We could reuse headers which would reduce a lot of size. In many ways this is a great idea, have a big linked list of notes containing links to headers, move frequently accessed entries to the front etc. However this means reusing message keys which is bad practice/NIST recommends against.
            \item Underlying symmetric/asymmetric schemes. Maybe I could have improved the block size but ECC patents mean its not really found in open libraries.
        \end{itemize}
	\item Key management and size (NIST recommendations).
        \item Conclusion : simple bcast encryption using AES and RSA underneath. However, modularity and extensibility would be useful because there are improvements. In particular ECC would give dramatically smaller overheads. 
    \end{itemize}

\section{Testing plan}
    \begin{itemize}
	\item What kinds of testing will I use?
	\item How can I make these tests possible?
	\item Test bed or framework that needs to be in place
        \begin{itemize}
            \item Need a method of simulating the Facebook JPEG compression process. Use libjpeg since it most closely matches the compression signature.
            \item Need a BER (bit error rate) calculator. Again coded as a C function.
            \item FireBug and FireUnit for unit testing and profiling JavaScript functions.
            \item gprof for profiling C/C++ functions.
        \end{itemize}
        \item Security testing \url{http://mtc-m18.sid.inpe.br/col/sid.inpe.br/ePrint 4080/2006/12.20.12.15/doc/v1.pdf}
        \begin{itemize}
            \item Threat analysis. Must compromise since full security audit beyond scope of project. Look only at UTF8 decoder and message retrieval process. As an extension expand threat model to include public key and image retrieval.
            \item Risk analysis
            \item Test plan elabouration
        \end{itemize}
    \end{itemize}
    
    
\section{Proffesional practice stuff}
    \begin{itemize}
	\item Software development model. Rapid application development. Create a rough protoype. Test, fix bugs, refactor, then rough out new functionality and repeat.
	\item Version/source control. Git and project locker.
	\item Performance bounds. Of what???
    \end{itemize}
    
\section{Requirements analysis}

    Broadly speaking we want the program to enable confidential sharing of messages and images, even given large scale adoption. It should also do this without negatively effecting the normal Facebook/browsing experience for both users and non-users. For users - this means no signifantly longer waiting times and security vulnerabilites. For non-users this mean no garbage posts filling up their newsfeeds.
    
    There are uncertainties and/or tradoffs associated with certain approaches to encryption and encoding data in images (and to a lesser extent error correction). It is also clear that in some cases the optimal approach is well beyond the scope of this project. Therefore, it is highly important that we adopt a modular structure that fascillitates switching between differing schemes and permits future extension.
    
    So from this we derive the requirements in detail:
    
    \begin{itemize}
    
        \item User requirements
        
        The user must be able to carry out the following tasks:
        \begin{itemize}
            \item Specify a group of recipients and carry out a broadcast-encrypted conversation. This should extend to (at least) status updates, posts and comments.
            \item Specify a group of recipients and upload a broadcast-encrypted image.
            \item Recover a broadcast-encrypted image.
            \item Move to another device and retain the ability to perform all of the above.
        \end{itemize}
        
        For non-users, interaction with application users should have minimal 
        
        \item Functional requirements
        
        \begin{itemize}
            \item Must meet security requirements - lets say ensures confidentiality from Facebook up till 2030. To meet this we would want a formal proof of the scheme and point to use of NIST recommendations for key size and management.
            \item Must not introduce security holes in the browser sandbox - though we can only satisfy this to an extent. This is a bad requirement, but we still need it really?
            \item Encryption, FEC and images - must be able to switch implementations and add new ones. If not at run time then at least at design time.
        \end{itemize}
            
        \item Performance requirements
        
        Due to limitations we only test on a single PC (my laptop). However this is acceptable since its only moderately specifed, point to some proof.
        
        \begin{itemize}
        
            \item User base
            \begin{itemize}
                \item Should work with recipient group sizes of up to 400.
            \end{itemize}
            
            \item Waiting times
            \begin{itemize}
                \item Encrypted message and images should load within acceptable time limits. Define acceptable as \url{http://www.useit.com/papers/responsetime.html}.
                
                \item Submission times for encrypted messages and images should be within acceptable limits. Define acceptable as \url{http://www.useit.com/papers/responsetime.html}.
            \end{itemize}
            
        \end{itemize}
    
    \end{itemize}