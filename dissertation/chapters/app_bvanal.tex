\chapter{Boundary-value analysis}
\label{app:bv}

For the UTF-8 decoder we derived test cases using boundary value analysis. Below is a complete specification of accepted and invalid inputs.

\begin{itemize}
        \item We accept any valid, non-overlong, UTF8 byte sequences, maximum length 4-bytes, with scalar value:
        \begin{itemize}
            \item 0xB0 - 0xD7FF
            \item 0xE000 - 0x100AF
            \item 0x1B000 - 0x1BFFE (would-be surrogate pairs)
            \item 0x10F0000 (indicates a padding byte was added, only one allowed per decode)
        \end{itemize}
        \item We therefore must throw an exception whenever a valid UTF8 byte sequence is presented with scalar value:
        \begin{itemize}
            \item 0x0 - 0xAF (out of range)
            \item 0xD800 - 0xDFFF (surrogate pair characters)
            \item 0x100B0 - 0x1AFFF (out of range)
            \item 0x1BFFF - 0x10EFFFF (out of range)
            \item 0x10F001 - 0x1FFFFF (out of range)
        \end{itemize}
        \item We also throw an exception for valid UTF8 sequences when:
        \begin{itemize}
            \item They have an overlong form i.e. the same scalar value can be represented using a shorter byte sequence.
            \item They have scalar value 0x10F0000 (padding character) but this has already been seen during decoding.
            \item They have scalar value 0x10F0000 (padding character) but the final decoded byte sequence (before padding removal) has length less than 2.
            \item The final decoded byte sequence has length less than 1.
            \item They are longer than 4-bytes.
        \end{itemize}
        \item Naturally we reject any (invalid) UTF8 byte sequences with:
        \begin{itemize}
            \item Unexpected continuation bytes when we expect a start character.
            \item A start character which is not followed by the appropriate amount of valid continuation bytes - including start characters right at the end of a sequence.
        \end{itemize}
    \end{itemize}
    
For sanitisation, we only permit alphanumeric characters along with the a small selection of symbols {\tt ,.?!()'*\&\%\$£}. Boundary-values can be easily computed from the ASCII character set.