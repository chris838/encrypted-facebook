\chapter{Threat analysis}
\label{app:threat}

\section{Identification of threats}

We ignore threats that would be present anyway (e.g. if HTTPS isn't used when browsing) and only consider additional threats introduced by the application. We take an attack centric approach and identify a number of possible threats:

\begin{desc}
    \item[Attack 1] Attacker breaks the encryption scheme by brute force methods.

    \item[Attack 2] Attacker gains access to user's computer through the download and execution of either a JPEG file or a public key file.

    \item[Attack 3] Attacker gains access to user's computer by injecting code into {\tt eval()} which runs outside the browser sandbox.

    \item[Attack 4] Attacker gains access to Facebook account through browser code injection, outside the sandbox. Typically this could involve cross-site scripting.

    \item[Attack 5] Attacker carries out middle-person attack by intercepting and switching public keys.

    \item[Attack 6] Attacker causes DoS by creating a malicious object. An example could be a self referential object, or an overly large object.
\end{desc}

\section{Risk analysis}

We define risk as $Vulnerability \times Threat \times Impact$ based on \cite{security}.

\begin{desc}
    \item[Attack 1] Medium impact (loss of privacy) but low vulnerability provided proper key lenths are used and proper protocols observed. Low risk.

    \item[Attack 2] Executing arbitrary code on the user's machine is the highest impact possible. However, vulnerability is low since getting code to be executed once downloaded would be difficult. Medium risk.

    \item[Attack 3] Again, executing arbitrary code on the user's machine is the highest impact possible. Vulnerability is medium. We identify this as a high risk attack.

    \item[Attack 4] This is medium impact (loss of privacy) but high vulnerability since code injection of user defined input is happening automatically, without mandate from the user. We identify this as a high risk attack.

    \item[Attack 5] Impact is medium (loss of privacy) and vulnerability is low since Graph API calls use HTTPS. Low risk.

    \item[Attack 6] Impact is low (loss of availability) but vulnerability is high, again due to automated decryption. Medium risk.
\end{desc}



\section{Proposed measures}

We propose a set of measures that are proportionate to risk.

\begin{desc}
    \item[Attack 1] Ensure key sizes are appropriate given the type of attacker. See Section \ref{ssec:keys}.

    \item[Attack 2] Ensure only legitimate JPEG and public key files are downloaded. Public key files can be vetted on their character set and size, JPEG's by their extension.

    \item[Attack 3] Wrap all calls to {\tt eval()} with a {\tt secureEval()} function which attempts to prevent malicious use. 

    \item[Attack 4] Sanitize all user inputs and ensure sanitisation can't be bypassed e.g. through the UTF-8 decoder\cite{utf8}. Also sanitize output whenever inserting code into the DOM.

    \item[Attack 5] Informing the user whenever public keys are updated removes a large amount of risk. Complete protection would be impractical without use of OOB channels.

    \item[Attack 6] This need not be a large problem since decryption is built to fail gracefully (see Section \ref{ssec:ident-targets}). Include some simple run time checks to limit iteration and special characters (combined with proper sanitisation) to prevent tags-within-tags.
\end{desc}

\section{Security testing plan}

We outline a partial plan for testing some of the proposed measures during development.

\begin{itemize}
    \item Testing of secureEval should ensure that only content in the repose text from the Facebook domain is parsed, and that the contents is a JSON object only.
    
    \item Testing of sanitisation and UTF-8 decoder can be performed using boundary variable analysis.
    
    \item Attempt at creating malicious objects can be made to test decoder.
\end{itemize}
        